\section{Hybrid Paper Recommenders} \label{sec:hybrid-recommenders-for-academic-papers}

\Cref{sec:citation-analysis} and \Cref{sec:semantic-analysis} discussed various existing citation-based and content-based approaches to recommender systems. While literature on citation analysis has grown steadily over the past few decades, the recent surge in \ac{LLMs} has spurred an explosion of research on content-based methods.
However, in spite of the comprehensive literature in both of these areas, there is limited research on \emph{hybrid} recommender systems that integrate both language models and citation-based approaches for recommending academic papers. This combination, though promising for more personalized and comprehensive recommendations, remains largely unexplored.

This section provides an overview of previous work on hybrid paper recommendation, identifies gaps in the existing literature, and emphasizes the contributions of this thesis to the field.


\subsection{Literature Overview}

Kanakia et al. \cite{KanakiaScalableHybrid2019} developed a hybrid recommender system that effectively integrates citation-based and content-based recommendation methods.
The authors contend that while pure citation-based recommendations deliver high quality, they face challenges with coverage.
Conversely, while content-based recommendations offer high coverage, they often lag in quality and demand significant computational resources.
To address these challenges, they introduced a weighted mixed hybridization approach combining the two methods.
The citation-based recommender employs co-citation analysis using citation data from the Microsoft Academic Graph\footnote{\url{https://www.microsoft.com/en-us/research/project/microsoft-academic-graph/}}, while the content-based recommender utilizes a Word2Vec model \cite{MikolovEfficientEstimation2013} to produce document embeddings from paper titles and abstracts.
To bypass the quadratic runtime complexity of pairwise similarity computations, they adopt spherical K-Means clustering to minimize the number of pairwise comparisons.

Färber and Sampath \cite{FarberHybridCiteHybrid2020} introduce a hybrid model for citation recommendation that incorporates multiple content-based approaches, encompassing embedding-based, topic modeling, and information retrieval techniques.
The authors focus on local citation recommendation, wherein the objective is to suggest citations for specific sections of text within a document, termed as citation contexts.
Their model utilizes embeddings to capture semantic similarity, topic modeling to identify shared themes, and information retrieval to pinpoint similar documents based on term frequency. By combining these methods into a hybrid system, they show that different similarity measures can provide complementary benefits in citation recommendation.

Khadka et al. \cite{KhadkaCapturingExploiting2020} designed a hybrid recommender system that blends \ac{CBF}, \ac{CF} techniques, and citation knowledge.
Their research stems from the realization that pure collaborative filtering methods are ineffective for recommending newly published papers, as they suffer from the cold-start problem.
As is common in \ac{CBF}, they initially build user profiles using \emph{user-based} features, such as the user's publication history. Then, citation knowledge is used to extract \emph{item-based} features of research papers, such as both direct and indirect citations, along with the citation context.
Recommendations are generated by computing similarity scores between users and papers based on the \ac{CF} nearest neighbor heuristic.

Finally, Ostendorff et al. \cite{OstendorffAspectbasedDocument2020} developed an aspect-based document similarity measure for academic papers by fusing content-based and citation-based approaches.
The authors observe that conventional document similarity measures only provide a binary sense of similarity, without explaining in what aspects two documents are similar.
To enhance the granularity of document similarity, they propose a pairwise document classification task using section titles in which citations occur as labels for the document pairs.
By applying various Transformer models to this task, they succeed in capturing the specific aspects in which two documents align.


\subsection{readnext: A Hybrid Recommender System for Computer Science Papers}

This thesis introduces the \emph{readnext} framework, a hybrid recommender system for Computer Science papers that incorporates both citation-based and content-based features, as well as global document characteristics.


\subsubsection*{On the shoulders of giants}

Our approach derives inspiration from previous work in the field. In line with the content-based approach of Nascimento et al. \cite{NascimentoSourceIndependent2011}, \emph{readnext} also uses a single query paper to generate a list of recommended papers.
Similar to the approach of Khadka et al. \cite{KhadkaCapturingExploiting2020}, we use co-citation analysis and bibliographic coupling to incorporate citation information into the recommendation process. We agree with the authors that many citation-based methods are biased towards established papers. Therefore, we include the publication date of papers as a feature to prioritize recently published papers in the recommendation process.

As noted by Beel et al. \cite{BeelResearchpaperRecommender2016}, TF-IDF remains a strong baseline for content-based recommendation, competitive with modern Transformer models. Therefore, we use TF-IDF as a baseline for our content-based recommender.
Due to the strong performance of BM25 for paper recommendation highlighted by Mohamed et al. \cite{A.MohamedBERTELMo2019}, we include BM25 as an alternative.
Moreover, BERT has repeatedly proven its capability within document retrieval tasks \cite{AkkalyoncuYilmazApplyingBERT2019,MassAdhocDocument2020}, while SciBERT has been shown to be particularly effective for scientific papers \cite{OstendorffAspectbasedDocument2020}. Thus, we include these two language models as options for our framework.
Finally, the work of Kanakia et al. \cite{KanakiaScalableHybrid2019} has influenced performance optimization strategies within \emph{readnext} to address the quadratic runtime complexity of pairwise similarity computations.


\subsubsection*{Contributions of this thesis}

However, the \emph{readnext} recommender system sets itself apart from existing work in several ways.

First, it provides unique customizability, allowing users to select their preferred language model for the content-based recommender and define the weighting scheme for combining global document features and citation features for the citation-based recommender.
Recommendations can be retrieved from each of the two individual recommenders as well as from the hybrid recommender for both hybrid orderings.

Second, this thesis presents a systematic and data-driven evaluation of recommendations generated by \emph{readnext} with different input parameters and recommender orderings.
The results allow for robust statements about the benefits of incorporating a content-based modeling step into traditional citation-based literature recommenders, offering insights into the interaction of these methods and suggesting directions for future research on hybrid recommender systems for academic papers.

Finally, \emph{readnext} is open-source and freely accessible on GitHub\footnote{\url{https://github.com/joel-beck/readnext}}. The project not only provides the source code to replicate the results of this thesis, but also includes comprehensive documentation\footnote{\url{https://joel-beck.github.io/readnext/}} on how researchers can install, extend and further customize the recommender system to their own needs. This transparency allows practitioners to use the \emph{readnext} framework as a foundation for their future research, advancing the field of hybrid recommender systems for academic papers.
